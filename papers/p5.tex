%\documentclass[aps,floatfix,prd,showpacs]{revtex4}
\documentclass[aps,floatfix,prd,showpacs,twocolumn]{revtex4}
\usepackage{graphicx}% Include figure files
\usepackage{dcolumn}% Align table columns on decimal point
\usepackage{bm}% bold math

\voffset 1.0cm

\begin{document}

\title{A Discourse-Aware Attention Model for
Abstractive Summarization of Long Documents}
\author{
Arman Cohan
Franck Dernoncourt
Doo Soon Kim
Trung Bui
Seokhwan Kim 
Walter Chang
Nazli Goharian†
}
\affiliation{
Department of Computer Science, 
Georgetown University, 
Washington, DC}

\date{}

\begin{abstract}
Neural abstractive summarization models have
led to promising results in summarizing relatively short documents. We propose the first
model for abstractive summarization of single,
longer-form documents (e.g., research papers).
Our approach consists of a new hierarchical
encoder that models the discourse structure of
a document, and an attentive discourse-aware
decoder to generate the summary. Empirical
results on two large-scale datasets of scientific
papers show that our model significantly outperforms state-of-the-art models. 
\end{abstract}


\section{Introduction}
Existing large-scale summarization datasets
consist of relatively short documents. For example, articles in the CNN/Daily Mail dataset (Hermann et al., 2015) are on average about 600 words
long. Similarly, existing neural summarization
models have focused on summarizing sentences
and short documents. In this work, we propose a
model for effective abstractive summarization of
longer documents. Scientific papers are an example of documents that are significantly longer
than news articles (see Table 1). They also follow a standard discourse structure describing the
problem, methodology, experiments/results, and
finally conclusions (Suppe, 1998).
Most summarization works in the literature
focus on extractive summarization. Examples
of prominent approaches include frequency-based
methods (Vanderwende et al., 2007), graph-based
methods (Erkan and Radev, 2004), topic modeling (Steinberger and Jezek, 2004), and neural
models (Nallapati et al., 2017). Abstractive summarization is an alternative approach where the
generated summary may contain novel words and
phrases and is more similar to how humans summarize documents (Jing, 2002). Recently, neural methods have led to encouraging results in
abstractive summarization (Nallapati et al., 2016;
See et al., 2017; Paulus et al., 2017; Li et al.,
2017). These approaches employ a general framework of sequence-to-sequence (seq2seq) models
(Sutskever et al., 2014) where the document is
fed to an encoder network and another (recurrent)
network learns to decode the summary. While
promising, these methods focus on summarizing
news articles which are relatively short. Many
other document types, however, are longer and
structured. Seq2seq models tend to struggle with
longer sequences because at each decoding step,
the decoder needs to learn to construct a context
vector capturing relevant information from all the
tokens in the source sequence (Shao et al., 2017).
Our main contribution is an abstractive model
for summarizing scientific papers which are an
example of long-form structured document types.
Our model includes a hierarchical encoder, capturing the discourse structure of the document and a
discourse-aware decoder that generates the summary. Our decoder attends to different discourse
sections and allows the model to more accurately
represent important information from the source
resulting in a better context vector. We also introduce two large-scale datasets of long and structured scientific papers obtained from arXiv and
PubMed to support both training and evaluating
models on the task of long document summarization. Evaluation results show that our method outperforms state-of-the-art summarization models1.


\section{Results}
Our main results are shown in Tables 2
and 3. Our model significantly outperforms the
state-of-the-art abstractive methods, showing its
effectiveness on both datasets. We observe that
in our ROUGE-1 score is respectively about 4 and
3 points higher than the abstractive model PntrGen-Seq2Seq for the arXiv and PubMed datasets,
providing a significant improvement. Our method
also outperforms most of the extractive methods
except for LexRank in one of the ROUGE scores.
We note that since extractive methods copy salient
sentences from the document, it is usually easier for them to achieve higher ROUGE scores.
Figure 2 illustrates the effectiveness of our
model extensions in capturing various discourse
information from the papers. It can be observed
that the state-of-the-art Pntr-Gen-Seq2Seq model
generates a summary that mostly focuses on introducing the problem, whereas our model generates
a summary that includes more information about
the methodology and impacts of the target paper.
This indicates that the context vector in our model
compared with Pntr-Gen-Seq2Seq is better able to
capture important information from the source by
attending to various discourse sections.


\section{Conclusions}
This work was the first attempt at addressing
neural abstractive summarization of single, long
documents. We presented a neural sequence-tosequence model that is able to effectively summarize long and structured documents such as scientific papers. While our results are encouraging,
there is still much room for improvement for this
challenging task; our new datasets can help the
community to further explore this problem.
We note that following the convention in the
summarization research, our quantitative evaluation is performed by ROUGE automatic metric.
While ROUGE is an effective evaluation framework, nuances in the coherence or coverage of the
summaries are not captured with it. It is non-trivial
to evaluate such qualities especially for long document summarization; future work can design expert human evaluations to explore these nuances.

\acknowledgments
We thank the three anonymous reviewers for
their comments and suggestions.

\begin{thebibliography}
Dzmitry Bahdanau, Kyunghyun Cho, and Yoshua Bengio. 2014. Neural machine translation by jointly
learning to align and translate. arXiv preprint
arXiv:1409.0473 .
Sumit Chopra, Michael Auli, Alexander M Rush, and
SEAS Harvard. 2016. Abstractive sentence summarization with attentive recurrent neural networks. In
HLT-NAACL. pages 93–98.
Arman Cohan and Nazli Goharian. 2015. Scientific article summarization using citation-context
and article’s discourse structure. In Proceedings of the 2015 Conference on Empirical Methods in Natural Language Processing. Association for Computational Linguistics, Lisbon, Portugal, pages 390–400. http://aclweb.org/
anthology/D15-1045.
Arman Cohan and Nazli Goharian. 2017a. Contextualizing citations for scientific summarization using
word embeddings and domain knowledge. arXiv
preprint arXiv:1705.08063 .

\end{thebibliography}

\end{document}